%IMPORTS
\documentclass[a4paper, 11pt]{article}
\usepackage[utf8]{inputenc} 
\usepackage[T1]{fontenc}
\usepackage[catalan]{babel}
\usepackage{amsmath, amssymb, amsthm}
\usepackage[margin=1in]{geometry}
\usepackage{enumerate}
\usepackage{array}
\usepackage{graphicx}
\usepackage{wrapfig}
\usepackage{ragged2e} 
\usepackage{subfig}
\usepackage{caption}
\usepackage{subcaption}
\usepackage[dvipsnames]{xcolor}
%\usepackage[table]{xcolor}
\usepackage{float}
\usepackage{chngcntr}
\usepackage{ragged2e}
\usepackage{multirow}
\usepackage{vmargin}
\usepackage{hyperref}
\usepackage{url}
\usepackage{fancyhdr}
\usepackage{bigints}
\usepackage{listings}
\usepackage{xcolor,colortbl}
\usepackage{booktabs}
%\usepackage{slashbox}
\definecolor{bluebell}{rgb}{0.64, 0.64, 0.82}
\definecolor{atomictangerine}{rgb}{1.0, 0.6, 0.4}
\definecolor{applegreen}{rgb}{0.55, 0.71, 0.0}
\definecolor{frenchblue}{rgb}{0.0, 0.45, 0.73}
\definecolor{darkpastelgreen}{rgb}{0.01, 0.75, 0.24}
\definecolor{darkpastelblue}{rgb}{0.47, 0.62, 0.8}
\definecolor{navy}{rgb}{0,0,128}
\definecolor{codegreen}{rgb}{0,0.6,0}
\definecolor{codegray}{rgb}{0.5,0.5,0.5}
\definecolor{codepurple}{rgb}{0.58,0,0.82}
\definecolor{backcolour}{rgb}{0.95,0.95,0.92}
\definecolor{amaranth}{rgb}{0.9, 0.17, 0.31}
\definecolor{GRAY}{rgb}{0.75, 0.75, 0.75}
\definecolor{deepfuchsia}{rgb}{0.76, 0.33, 0.76}
\definecolor{deepmagenta}{rgb}{0.8, 0.0, 0.8}
\definecolor{funcblue}{rgb}{0.36, 0.57, 0.9}
\lstdefinestyle{mystyle}
{language=c,
    backgroundcolor=\color{white},   
    commentstyle=\color{codegreen},
    keywordstyle=\color{RoyalBlue},
    numberstyle=\tiny\color{codegray},
    stringstyle=\color{codepurple},
    basicstyle=\ttfamily\footnotesize,
    breakatwhitespace=false,         
    breaklines=true,                 
    captionpos=b,                    
    keepspaces=true,                 
    %numbers=left,                    
    numbersep=5pt,                  
    showspaces=false,                
    showstringspaces=false,
    showtabs=false,                  
    tabsize=2
}
\lstdefinestyle{Bash}
{language=bash,
keywordstyle=\color{blue},
basicstyle=\ttfamily,
morekeywords={peter@kbpet},
morekeywords=[2]{make},
keywordstyle=[2]{\color{blue}},
literate={\$}{{\textcolor{blue}{\$}}}1 
         {:}{{\textcolor{blue}{:}}}1
         {~}{{\textcolor{blue}{\textasciitilde}}}1,
}
\lstdefinestyle{BASH}
{language=bash,
keywordstyle=\color{blue},
basicstyle=\ttfamily,
morekeywords={peter@kbpet},
morekeywords=[2]{make},
fontsize=5pt
keywordstyle=[2]{\color{blue}},
literate={\$}{{\textcolor{blue}{\$}}}1 
         {:}{{\textcolor{blue}{:}}}1
         {~}{{\textcolor{blue}{\textasciitilde}}}1,
}

\definecolor{mGreen}{rgb}{0,0.6,0}
\definecolor{mGray}{rgb}{0.5,0.5,0.5}
\definecolor{mPurple}{rgb}{0.58,0,0.82}
\definecolor{backgroundColour}{rgb}{0.95,0.95,0.92}

\lstdefinelanguage{GERONA}{
    classoffset = 1,
    morekeywords = {for, if},
    keywordstyle = \color{atomictangerine},
    classoffset = 2,
    alsoletter=\#,
    morekeywords = {\#pragma, omp, parallel, ordered},
    keywordstyle = \color{bluebell},
    classoffset = 0,
    sensitive = true,
    morecomment = [l]{//},
    morecomment = [s]{/*}{*/},
    morecomment = [s]{/**}{*/},
    commentstyle = \color{applegreen},
    morestring = [b]",
    morestring = [b]',
}

\lstdefinestyle{CStyle}{
    %backgroundcolor=\color{backgroundColour},   
    commentstyle=\color{mGreen},
    keywordstyle=\color{magenta},
    numberstyle=\tiny\color{mGray},
    stringstyle=\color{mPurple},
    basicstyle=\footnotesize,
    breakatwhitespace=false,         
    breaklines=true,                 
    captionpos=b,                    
    keepspaces=true,                 
    numbers=left,                    
    numbersep=5pt,                  
    showspaces=false,                
    showstringspaces=false,
    showtabs=false,                  
    tabsize=2,
    language=GERONA
}

\lstset{style=CStyle}


\setpapersize{A4}
\setmargins{2.5cm}     % margen izquierdo
{2.6cm}                % margen superior
{16.5cm}               % anchura del texto
{23.7cm}               % altura del texto
{10pt}                 % altura de los encabezados
{0cm}                  % espacio entre el texto y los encabezados
{0pt}                  % altura del pie de página
{1cm}                  % espacio entre el texto y el pie de página
\renewcommand{\baselinestretch}{1.5}
\begin{document}

\begin{titlepage}
    \centering
    {\bfseries\LARGE \hspace{1.9em} Universitat Autònoma de Barcelona\newline Facultat de Ciències\par}
    \vspace{2cm}
    {\hspace{-1em}\includegraphics[width=0.7\textwidth]{MatCAD3.jpg}\par}
    \vspace{1cm}
    {\scshape\Huge Pràctica 1\par} 
    \vspace{1cm}
    {\Large \itshape Autors: \par}
    {\Large \hspace{-1.75em} Gerard Lahuerta \& Ona Sánchez \par}
    {\Large 1601350 --- 1601181 \par}
    \vspace{1cm}
    {\Large 15 de Octubre del 2022\par}
\end{titlepage}

\justifying

\newpage
\section{Introducció}
L'objectiu d'aquesta pràctica és optimitzar el temps d'execució del programa \textcolor{frenchblue}{energystorm}, programat al fitxer \textcolor{darkpastelgreen}{energy\_storm.c}. \\
Per tal d'optimitzar el funcionament del programa, es recurrirà a la paralel·lització de totes les instruccions repetitives, sempre i quan sigui possible, que generin un cost computacional elevat, al fitxer \textcolor{darkpastelgreen}{energy\_storm\_omp.c}. \\
\section{Anàlisi del problema}
A l'estudi realitzat anteriorment sobre els temps de diversos bucles del programa\footnote{L'estudi esmentat és el ja entregat anteriorment, si es vol consultar és: \textcolor{blue}{\href{https://drive.google.com/file/d/1cSu44VuoF-0nm9SR1Bhz5VClLV0hwk4E/view?usp=sharing}{Estudi Pràctica 1 CAP}}}, es va veure que la fase que més tarda en executar-se en la simulació és el bucle iniciat a la linia \textbf{183}, ja que el percentatge de temps que s'hi inverteix és casi del 100\% del total del temps d'execució, pel que els esforços en optimitzar el codi haurien d'anar enfocats a aquest bucle.\\\\
Per altra banda, també es va concloure que el bucle iniciat a la línia \textbf{198} podria ser paral·lelitzat, pel que s'ha d'estudiar quina és la millor manera d'optimitzar-lo.\\\\
S'observen també bucles diversos que podrien unir-se en un de sol, com és el cas dels fors de les línies \textbf{175} i \textbf{176} (ja que usen el mateix rang de valors per l'índex k i no depenen entre ells), es procedirà, també, a paral·lelitzar aquest nou bucle conjunt per agilitzar els càlculs quan s'ha de fer moltes iteracions del mateix (ja que depen el nobre d'iteracions en funció del nombre de cèlul·les que hi introduïm).\\
\section{Disseny de la solució}
Expossem i expliquem ara les seccions del codi modificades:
\begin{lstlisting}[language = GERONA, firstnumber = 178]
#pragma omp parallel for 
for( k=0; k<layer_size; k++ ){
    layer[k] = 0.0f;
    layer_copy[k] = 0.0f;
}
\end{lstlisting}
\begin{lstlisting}[language = GERONA, firstnumber = 200]
#pragma omp parallel for 
for( k=0; k<layer_size; k++ ) {
    /* Update the energy value for the cell */
    update( layer, layer_size, k, position, energy );
}
\end{lstlisting}
\begin{lstlisting}[language = GERONA, firstnumber = 209]
#pragma omp parallel for 
for( k=0; k<layer_size; k++ )
    layer_copy[k] = layer[k];
\end{lstlisting}
\begin{lstlisting}[language = GERONA, firstnumber = 228]
#pragma omp parallel for ordered
for( k=1; k<layer_size-1; k++ ) {
    /* Check it only if it is a local maximum */
    if ( layer[k] > layer[k-1] && layer[k] > layer[k+1] ) {
        #pragma omp ordered
        if ( layer[k] > maximum[i] ) {
            maximum[i] = layer[k];
            positions[i] = k;
        }
    }
}
\end{lstlisting}
S'ha simplificat la paral·lelització de la millor manera possible per tal d'optimitzar al màxim el codi.\\
En els tres primers blocs, com el treball és igual independentment de la iteració en que es trobi el bucle i no hi ha problemes de concurrència, s'ha pogut utilitzar la implementació més senzilla pels bucles de la que disposa l'OpenMP: \textit{\#pragma omp parallel for}.\\\\
Per altre banda, en l'últim bloc, degut a que es busca el màxim i la posició d'aquest valor la llista \textit{layer}, s'ha hagut de recorrir a la utilització de la clàusula \textit{orderer}, ja que la clàusula reduction (si ve ens permetia obtenir el màxim) no ens permetia obtenir les dues informacions que necessitàvem.\\
Tot i així, la implementació és suficienment bona, ja que només recorrim a ella una vegada la condició del primer for es compleix, pel que una gran quantitat de valors no accedeixen a aquesta zona no paral·lelitzada del for.\\\\
En cas de voler consultar les zones modificades, en les seccions de codi hi ha indicades les files on es troben les comandes al fitxer \textcolor{darkpastelgreen}{energy\_storm\_omp.c}.
\section{Resultat}
\begin{table}[h]
  \centering
  \begin{tabular}{l||c||c|c|c|c}
        \multirow{3}{*}{\cellcolor{black}{}} & \multicolumn{5}{c}{\textbf{TEMPS}}\\\hline
        \cellcolor{black}{} & \multirow{2}{*}{SEQÜENCIAL} & \multicolumn{4}{c}{PARAL·LELITZAT}\\ 
        \cellcolor{black}{} &  & 2 fluxos & 4 fluxos & 6 fluxos & 12 fluxos\\\hline\hline
        Test 2 & 43.369 & 22.099 & 11.168 & 5.870 & 4.194 \\\hline
        Test 7 & 241.569 & 122.442 & 61.390 & 30.969 & 20.755\\\hline
        Test 8 & 5.260 & 3.120 & 2.038 & 1.404 & 1.275\\\hline\hline
        \cellcolor{black}{} & \multicolumn{5}{c}{\textbf{Acceleració}} \\\hline\hline
        Test 2 & - & 1.962 & 3.883 & 7.388 & 10.340 \\\hline
        Test 7 & - & 1.972 & 3.934 & 7.800 & 11.639\\\hline
        Test 8 & - & 1.685 & 2.580 & 3.746 & 4.125\\
    \end{tabular}
\end{table}
\hspace{-1.5 em}S'observa com s'ha optimitzat el programa obtenint una millora de fins a $11.639$ vegades més ràpid amb $12$ fluxos en el test $7$.\\
Concluïm que hem assolit els objectius millorant zones que inicialment no havíem considerat (com el bucle de la línia $198$) i millorant les que ja s'havien estudiat anteriorment.
\section{Principals problemes}
Enumerem ara els principals problemes que han aparegut en el nostre procés d'optimitzacdió del codi:
\begin{enumerate}
    \item Optimització del bucle de la linia $200$: \\
    Inicialment voliem paral·lelitzar el bucle que el contenia pero per problemes amb dades, que estaven correlacionades, ens era impossible pel que és va decidir paral·lelitzar aquest per millorar el temps d'execució.
    \item Optimització del bucle de la linia $228$:\\
    Inicialment es va probar a optimitzar el bucle utilitzant clàusules com \textit{private}, \textit{lasprivate} i \textit{reduction} però no aconseguiem obtenir la posició màxima; pel que es va recurrir (eventualment) a la clàusula \textit{orderer} per a poder obtenir una millora del temps d'execució crivant valors que no complien el primer requisit (amb l'if del primer for) i a la vegada, executant de manera seqüencial el procès d'obtenció del màxim per, així, obtenir els dos valors requerits.
\end{enumerate}
\end{document}
