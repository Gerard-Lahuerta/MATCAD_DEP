%IMPORTS
\documentclass[a4paper, 11pt]{article}
\usepackage[utf8]{inputenc} 
\usepackage[T1]{fontenc}
\usepackage[catalan]{babel}
\usepackage{amsmath, amssymb, amsthm}
\usepackage[margin=1in]{geometry}
\usepackage{enumerate}
\usepackage{array}
\usepackage{graphicx}
\usepackage{wrapfig}
\usepackage{ragged2e} 
\usepackage{subfig}
\usepackage{caption}
\usepackage{subcaption}
\usepackage[dvipsnames]{xcolor}
%\usepackage[table]{xcolor}
\usepackage{float}
\usepackage{chngcntr}
\usepackage{ragged2e}
\usepackage{multirow}
\usepackage{vmargin}
\usepackage{hyperref}
\usepackage{url}
\usepackage{fancyhdr}
\usepackage{bigints}
\usepackage{listings}
\usepackage{xcolor,colortbl}
\usepackage{enumitem}
% \usepackage[english, spanish]{babel}
% \usepackage{slashbox}
\usepackage{diagbox}
\usepackage{eurosym}
\usepackage{relsize}

\begin{document}

\begin{titlepage}
    \centering
    {\bfseries\LARGE Universitat Autònoma de Barcelona\newline Facultat de Ciències\par}
    \vspace{2cm}
    {\hspace{-1em}\includegraphics[width=0.4\textwidth]{logo.png}\par}
    \vspace{1cm}
    {\scshape\Huge Pràctica 1\par} 
    \vspace{1cm}
    {\Large \itshape Autors: \par}
    {\Large  Gerard Lahuerta \par}
    {\Large 1601350 \par}
    \vspace{1cm}
    {\Large 8 maig 2022\par}
\end{titlepage}

\justifying

\newpage
\setcounter{page}{2}
\pagestyle{plain}
\tableofcontents
\cleardoublepage
\addcontentsline{}{chapter}{}
\newpage
\section{Exercici 1}
\subsection{Enunciat}
“Anxoves l’Escala” produeix llaunes d’anxoves i de sardines. Es planifica la producció mensualment.\\
La fàbrica té dues màquines d’enllaunar que proporcionen 300 hores de funcionament. L’empresa fa tests de qualitat electrònics a totes les seves llaunes. El proper mes hi ha disponibles
640 hores de test electrònic.\\
L’empresa té problemes de tesoreria i ha de limitar per al proper mes el pressupost de compra
de peix a 56000\euro\hspace{0.0625em} i de compra de llauna a 140000\euro.\\
La taula següent recull informació sobre la fabricació (envasat en unitats per hora, test en
unitats per hora, preu del peix per unitat, preu material llauna per unitat, i benefici net per
unitat). La unitat és una llauna.\\
\begin{table}[h]
    \centering
    \begin{tabular}{ c | c  c  c  c  c }
                 & envasat (un./h.) & test (un./h.) & preu peix & preu llauna & benefici \\ \hline
        anxoves  & 1600 & 800 & 0.30 & 0.11 & 0.26 \\
        sardines & 2000 & 800 & 0.20 & 0.08 & 0.20 \\
    \end{tabular}
\end{table}
\begin{enumerate}[label=(\alph*)]
    \item Quin hauria de ser el pla de fabricació del proper mes? Quin seria el benefici?
    \item L’empresa també vol saber:
    \begin{enumerate}
        \item[b.1] Donaria beneicis incrementar el temps de control? Quins?
        \item[b.2] Què passa si incrementem el pressupost de material de llauna?
        \item[b.3] Què passa si incrementem el pressupost de peix?
        \item[b.4] Es pot augmentar el temps de les màquines d’enllaunat a un cost de 100 eper hora. Seria recomanable fer aquest augment?
        \item[b.5] Seria recomanable augmentar el temps de test a un cost de 100 eper hora? 
    \end{enumerate}
    \item Si ajuntem els pressupostos de compra de material de la llauna i de peix podríem tenir
més benefici?
\end{enumerate}
\newpage
\subsection{Resolució del exercici}
\subsubsection{Apartat a)}
Sigui la funció de beneficis a maximitzar $B$, mitjançant l'enunciat, podem definir-la com:
$$B = \text{benefici\_peix} + \text{benefici\_anxoves}$$
Definim ara els components de la funció:
\begin{itemize}
    \item \textbf{benefici\_anxoves:}\\
    L'empresa obté un benefici net de la venta de llaunes d'anxoves ($B_A$) de $0.26$\euro.\\
    Anomenem a la quantitat de llaunes anxoves que fabriquem com $Q_A$.\\
    Podem deduir doncs l'expressió del benefici que obté la venta d'anxoves:
    $$benefici\_anxoves = B_A \cdot Q_A = 0.26 \cdot Q_A$$
    
    \item \textbf{benefici\_sardines:}\\
    L'empresa obté un benefici net de la venta de llaunes de sardines ($B_S$) de $0.2$\euro.\\
    Anomenem a la quantitat de llaunes de sardines que fabriquem com $Q_S$.\\
    Podem defuïr doncs l'expressió del benefici que obté la venta de sardines:
    $$benefici\_sardines = B_S \cdot Q_S = 0.2 \cdot Q_S$$
\end{itemize}
Per tant, la nostra funció de benefici $B(Q_i)$, per a $i\in\{A,S\}$, a maximitzar és:
\begin{center}
\fcolorbox{black}{white}{$B(Q_A,Q_S) = 0.26 \cdot Q_A + 0.2 \cdot Q_S$}
\end{center}
Trobem ara les restriccions:
\begin{itemize}
    \item [$ $] \textbf{Restriccions sobre la compra d'anxoves i sardines:}\\
    Tenim un pressupost màxim per a la compra dels dos tipus de productes (Anxoves ($A$) i Sardines ($S$)).\\
    La compra del prodcute és necesaria per a obtenir beneficis.\\
    De manera que la restricció sobre la compra del peix és:
    $$ 0 < P_{AS} = P_A \cdot Q_A + P_S \cdot Q_S \leq P_{max} $$
    On $P_I$ és el preu del producte $I$, tal que $I\in\{A,S\}$, i $P_{max}$ és el pressupost màxim en la compra dels dos tipis de peix.    
    \item [$ $] \textbf{Restriccions sobre la compra de llaunes:}\\
    Tenim un pressupost màxim per a la compra dels dos tipus de llaunes.\\
    La compra de les llaunes és necesaria per al enllaunament del produte.\\
    De manera que la restricció sobre les llaunes comprades és:
    $$ 0 < PL_{AS} = PL_A \cdot Q_A + PL_S \cdot Q_S \leq PL_{max} $$
    On $PL_I$ és el preu de les llaunes del producte $I$, tal que $I\in\{A,S\}$, i $PL_{max}$ és el pressupost màxim en les llaunes que disposem.    
    \item [$ $] \textbf{Restriccions sobre l'ús de les màquines d'enllaunament:}\\
    L'empresa disposa de dues màquines encarregades d'enllaunar les sardines i les anxoves.\\
    Cada màquina pot treballar en paral·lel, pel que cada una de les màquines ahura de fer la mitad del treball.\\
    Podem deduïr que el nombre d'unitat de llaunes de peix (en total) estarà acotada per el el temps que es triga en produir-les de manerà que:
    $$0 < \frac{TE_{AS}}{2} = \frac{E_A^{-1}\cdot Q_A + E_S^{-1}\cdot Q_S}{2} \leq TE_{max}$$
    O el que és equivalent:
    $$0 < \frac{Q_A}{E_A}  + \frac{Q_S}{E_S} \leq 2\cdot TE_{max}$$
    On $E_I$ és el nombre d'unitats que s'enllaunen en una hora del producte $I$, tal que $I\in\{A,S\}$, i $TE_{max}$ és el temps màxim que pot treballar una  de les màquines d'enllaunament.
    \item [$ $] \textbf{Restriccions sobre l'ús de la màquina de test de qualitat:}
    La restricció de la màquina de test de qualitat de les llaunes és analoga a la de les màquines d'enllaunament.\\
    La diferencia és que només disposem d'una sola màquina per a fer els tests.\\
    Deduïm que la restricció serà:
    $$0 < T_{AS} = \frac{Q_A}{T_A} + \frac{Q_S}{T_S} \leq TT_{max}$$
    On $T_I$ és el nombre d'unitats que és testegen en una hora del producte $I$, tal que $I\in\{A,S\}$, i $TT_{max}$ és el temps màxim que pot treballar la màquina de testeig.
\end{itemize}
Finalment podem expressar les restriccions (substituïnt pels paràmetres coneguts):
\begin{enumerate}
    \item $ 0 \leq Q_S, Q_A $
    \item $ 0 < 0.3 \cdot Q_A + 0.2 \cdot Q_S \leq 56000 $
    \item $ 0 < 0.11 \cdot Q_A + 0.08 \cdot Q_S \leq 140000 $
    \item $0 < \frac{Q_A}{1600}  + \frac{Q_S}{2000} \leq 600$
    \item $0 < \frac{Q_A + Q_S}{800}  \leq 640 $
\end{enumerate}

\newpage
\hspace{-1.5em}Una vegada ja hem plantejat el problema el resolem mitjançant el programa $ glpk$.\\
El resultat obtingut per part del programa és:
\begin{table}[h]
    \centering
    \begin{tabular}{c|c}
        \textbf{Producte} & \textbf{Unitats ha comprar} \\ \hline
        Anxoves & $0$\\ \hline
        Sardines & $280000$\\ 
    \end{tabular}
\end{table}\\
És a dir, que per a obtenir el màxim benefici possible cal només produir Sardines; més concretament cal produir $280000$ llaunes de Sardines.\\
Aquest pla de fabricacció permet obtenir el màxim benefici per valor de $56000$\euro.
\subsubsection{Apartat b)}
Per a resoldre tots el subapartats de l'apartat \textit{(b)}, mostrem ara una part de l'informació obtinguda mitjançant la comanda \texttt{-o} del $glpk$.\\
\begin{table}[h]
    \centering
    \begin{tabular}{ c | c | c | c | c | c | c }
        No. & Row name & St & Activity & Lower bound & Upper bound & Marginal \\\hline\hline
        1 & benefici & B & $56000$ & & &  \\ \hline 
        2 & pressupost & NU & $56000$ & & $56000$ & $1$ \\\hline
        3 & pressupost\_llaunes & B & $22400$ & & $140000$ & \\\hline
        4 & temps\_enllaunament & B & $140$ & & $600$ & \\\hline
        5 & temps\_test & B & $350$ & & $640$ & \\
    \end{tabular}
    \caption{Taula extreta amb la comanda \texttt{-o} del $glpk$}
    \label{tab:my_label}
\end{table}
\begin{enumerate}
    \item \textbf{Donaria beneficis incrementar el temps de control? Quin?}\\
    Observem que de les $640$ hores màximes que pot treballar la màquina encarregar de fer el testeig, aquesta només treballa $350$.\\
    Per tant, incrementar les hores que pot fer tests la màquina no influiria en obtenir major benefici.\\
    \textit{El temps de treball de la màquina de testeig  és una restricció inactiva.}\\
    Aquesta informació l'hem obtinguda de la fila numero 4, ja que la variable \textit{temps\_test} és la que recull el nombré d'hores que s'utilitza la màquina de testeig per a obtenir el màxim benefici.
    \item \textbf{Què passa si incrementem el pressupost de material de llauna?}
    De la mateixa manera que amb la pregunta anterior, amb l'informació de la taula (més concretament la de la fila número 3, la fila dedicada a la variable pressupost\_llaunes), no utilitcem la totalitat del pressupost.\\
    Per tant, incrementar el pressupost no afectaria en obtenir més beneficis.\\
    \textit{El pressupost destinat a les llaunes és una restricció inactiva.}
    \item \textbf{Què passa si incrementem el pressupost de peix?}\\
    Mitjançant l'informació de la fila de pressupost (fila numero 1 corresponent al pressupost per a la compra de peix) deduïm que, exectivament, si l'augmentem el benefici seria major. Aixó es degut a que si utilitcem la totalitat del pressupost i, a més, el seu marginal és $1$; és a dir, que si augmentem en 1 unitat el pressupost obtindrem 1 unitat més de benefici (en aquest cas 1 unitat es 1\euro). \\
    \textit{El pressupost destinat a la compra de peix és una restricció activa.}
    \item \textbf{Es pot augmentar el temps de les màquines d’enllaunat a un cost de 100\euro \hspace{0.0625}per hora. Seria recomanable fer aquest augment?}\\
    Analogament amb la màquina de testeig de les llaunes, les hores màximes de treball de les màquines d'enllaunament ($300$ hores per $2$ màquines; és a dir, $600$ hores en total) no són utilitzades en la seva totalitat, només son utilitzades $150$.\\
    Per tant, incrementar el seu servei d'hores (independentment del cost que precissi l'augment) és innecessari per a incrementar els beneficis.\\
    \textit{El temps de treball de la màquina d'enllaunament és una restricció inactiva.}\\
    La informació per a la conclusió s'ha extret de la fila 3, dedicada a la variable temps\_enllaunament.
    \item \textbf{Seria recomanable augmentar el temps de test a un cost de 100\euro \hspace{0.0625} per hora?}\\
    Tal com hem explicat a l'apartat \textit{(b.1)}, el temps de testeig no influeix en els beneficis.\\
    Per tant, no seria recomanable augmentar el temps de testeg per $100$\euro per hora
\end{enumerate}
\subsubsection{Apartat c)}
Mitjançant les observacions fetes a l'apartat anterior i la informació que hi conté la taula, podem afirmar que si ajuntem els pressupostos de les llaunes amb els dels peixos tindrem més benefici.\\
Aixó esdevé a que no s'utilitza la totalitat del pressupost destinat a les llaunes, pel que part d'aquest pressupost no utilitzat pot ser aprofitat en augmentar el nombre de peixos a comprar per, així, fabricar més llaunes.\\
Com que el marginal del pressupost de peix és positiu, ens indica que si augmentem el pressupost el benefici ha de creixer també (en principi).
Per asegurar executem el nostre programa $glpk$ per a confirmar l'hipotesis.\\
Inserim aquí els resultats obtinguts:
\begin{table}[h]
    \centering
    \begin{tabular}{c|c}
        \textbf{Producte} & \textbf{Unitats ha comprar} \\ \hline
        Anxoves & $404923$\\ \hline
        Sardines & $107077$\\ 
    \end{tabular}
\end{table}\\
És a dir, si subtituïm en la formula obtinguda a l'apartat \textit{(a)} per als beneficis, trobem que el benefici serà d'uns $126695.38$ \euro. 







\newpage
\section{Exercici 2}
\subsection{Enunciat}
Dues ciutats generen residus i els seus residus s’envien a incineradores per a la seva combustió.\\
La producció diària de residus i les distàncies entre les ciutats i les incineradores són:
\begin{table}[h]
    \centering
    \begin{tabular}{ c | c | c  c }
         & residus produïts  & \multicolumn{2}{ c }{dist. a l’incineradora}  \\ 
         & (tones/dia) & A (en km) & B (en km) \\ \hline
        ciutat 1 & 500 & 30 & 20 \\
        ciutat 2 & 400 & 36 & 42 
    \end{tabular}
\end{table}\\
La incineració redueix cada tona de residus a 0.2 tones de deixalles, que s’han d’abocar en un
dels dos abocadors (Nord i Sud). Costa 3\euro\hspace{0.0625em} per km transportar una tona de material (ja siguin
residus o cendres). A continuació es mostren les distàncies (en km) entre les incineradores i
els abocadors:
\begin{table}[h]
    \centering
    \begin{tabular}{ c | c | c | c  c }
         & capacitat  & cost incineració & \multicolumn{2}{ c }{dist. als abocadors}  \\
         & (tones/dia) & euros/tona & Nord (en km) & Sud (en km) \\ \hline
         incineradora A & 500 & 40 & 5 & 8 \\
         incineradora B & 600 & 30 & 9 & 6 
    \end{tabular}
\end{table}
\begin{enumerate}[label=(\alph*)]
    \item Planteja un problema de programació lineal que proporcioni el pla per tal de minimitzar
el cost total de l’eliminació dels residus de les dues ciutats.

    \item Resol el problema en glpk. Digues quin és exactament el pla de transport i quin és el
cost total.
    \item Suposem que el transport entre les ciutats i les incineradores està limitat a 300 tones per
ruta.
\begin{enumerate}
        \item[c.1] Com hauríem d’organitzar el transport?
        \item[c.2] Observant el fitxer de solució que es genera amb glpk (opció -o), de quina ruta
valdria la pena ampliar-ne la capacitat? quin seria el decreixement del cost per
cada tona que s’ampliés?
        \item[c.3] Si calculem el cost per a aquesta ampliació de capacitat (només en una ruta), quadra
amb el que observem al fitxer generat amb l’opció -o? Per què?
    \end{enumerate}
    \item Què passaria si poséssim les incineradores a les ciutats? Cal tenir en compte que les
ciutats estan a una distància de 30 km entre elles i que aleshores les distàncies de les
incineradores als abocadors serien
\begin{table}[h]
    \centering
    \begin{tabular}{ c | c | c  c }
         & \multicolumn{2}{ c }{dist. als abocadors}  \\ 
         & Nord (en km) & Sud (en km) \\ \hline
        incineradora A & 35 & 38  \\
        incineradra B & 51 & 48  
    \end{tabular}
\end{table}\\
Calcula quina seria la reducció de cost total de transport.
\end{enumerate}
\newpage
\subsection{Resolució del exercici}
\subsubsection{Apartat a)}
Sigui la funció de cost a minimitzar $f(x)$, mitjançant l'enunciat, podem definir-la com:
$$f =   \text{cost\_transport\_ciutats}  + \text{cost\_incineracio}  +  \text{cost\_transport\_abocadors} $$
 Definim ara els components de la funció:
 \begin{itemize}
     \item \textbf{cost\_transport\_ciutats:}\\
     Les ciutats només poden accedir a dos incineradors; pel que si $T_{IJ}$ són les toneades desplaçades de la ciutat $I$ a la incineradora $J$ (essent $I\in \{1,2\}$ i $J \in \{A,B\}$), ens queda que:\\
     $$PT\cdot \left( \mathlarger{\mathlarger{\sum}}_{I\in \{1,2\}} \mathlarger{\mathlarger{\sum}}_{J \in \{A,B\}} T_{IJ} \cdot d_{IJ}\right) $$
     on $d_{IJ}$ és la distància entre la ciutat $I$ i la incineradora $J$ i $PT$ és el preu de transport per $km$ i per tonelada.\\
     Mitjançant l'informació de l'enunciat, substituïm els valors:
     $$cost\_transport\_ciutats = 3\cdot( 30\cdot T_{1A} + 20\cdot T_{1B} + 36\cdot T_{2A} + 42\cdot T_{2B})$$
     \item \textbf{cost\_incineracio:}\\
     El cost d'incineració per tona de material de cada incineradora varia, pel que definim $IN_J$ com el preu per tonelada d'incineració de la incineradora $J$, tal que $J\in\{A,B\}$.\\
     Si definim també com $T_J$ les tonelades que rep l'incineradora $J$, obtenim la següent expressió:
     $$\mathlarger{\mathlarger{\sum}}_{J \in \{A,B\}} T_J\cdot IN_J$$
     Com el que rep l'incineradora $J$ és la suma de les tonelades enviades a ella de les dues ciutats, podem expressar el cost d'incineració com:\\
    $$cost\_incineracio = 40\cdot(T_{1A} + T_{2A}) + 30\cdot(T_{1B}+T_{2B})$$
     \item \textbf{cost\_transport\_abocadors:}\\
     El cost de transport de les incineradores als abocadors és anàleg al de les ciutats als incineradors. La diferencia és que les incineradores redueïxen els residus al $20\%$ del que eren (els multipliquen per un factor de $0.2$).\\
     Per tant, si definim com $T_{JK}$ les tonelades que s'envien de la incineradora $J$ a l'abocador $K$ (essent $J \in \{A,B\}$ i $k \in \{N,S\}$, obtenim:\\
     $$PT\cdot \left( \mathlarger{\mathlarger{\sum}}_{J\in \{A,B\}} \mathlarger{\mathlarger{\sum}}_{K\in\{N,S\}} T_{JK} \cdot d_{JK}\right) $$
     on $d_{JK}$ és la distància entre la incineradora $J$ i l'abocador $K$ i $PT$ és el preu de transport per $km$ i per tonelada. Substituïnt, obtenim:\\
     $$cost\_transport\_abocadors = 3\cdot0.2 \cdot (5\cdot T_{AN} + 8\cdot T_{AS} + 9\cdot T_{BN}+ 6\cdot T_{BS})$$
 \end{itemize}
 Podem ara definim la funció de cost a minimitzar de la següent forma:
 \begin{equation*}
 \fcolorbox{black}{white}{
     \begin{split}
        f(T_{IJ},T_{JK}) = & \hspace{0.4em} 3\cdot( 30\cdot T_{1A} + 20\cdot T_{1B} + 36\cdot T_{2A} + 42\cdot T_{2B})  +\\
        &  + 40\cdot(T_{1A} + T_{2A}) + 30\cdot(T_{1B}+T_{2B})  + \\
        &  +  3\cdot0.2 \cdot (5\cdot T_{AN} + 8\cdot T_{AS} + 9\cdot T_{BN}+ 6\cdot T_{BS}) 
     \end{split}
     }
 \end{equation*}\\
 Trobem ara les restriccions:
 \begin{itemize}
     \item \textbf{Restriccions sobre les ciutats:}\\
     Cada ciutat te un nombre de residus que ha de transportar a les incineradores.\\
     La ciutat s'ha de quedar sense residus.\\
     Per tant, les tonelades de residus que es transporta entre una ciutat i una incineradora ha de ser extricament major o igual a $0$ i menor o igual al valor de residus que té:
     $$ 0 \leq T_{IJ} \leq Res_I$$
     on $Res_I$ són les tonelades de residus de la ciutat $I$, tal que $I\in{1,2}$.
     \item \textbf{Restriccions sobre les incineradores:}\\
     Cada incineradora té un màxim de residus que pot tractar; pel que la suma de residus enviat per les dues ciutats no pot suèrar aquest límit.\\
     Les incineradores no poden portat residus (cendres) a les ciutats.\\
     Per tant, les tonelades de residus que rep les incineradores ha de ser estrictament major o igual a $0$ i menor o igual al seu màxim:
     $$0 \leq T_J = T_{1J}+T_{2J} \leq Max_J$$
     on $Max_j$ són les tonelades de residus màximes que por tractar l'incineradora $J$, tal que $J\in{A,B}$.
     \item \textbf{Restriccions sobre els abocadors:}\\
     Els abocadors no tenen límitació de capacitat de tractament de residus.\\
     Les incineradores han de enviar els residus que tenen als abocadors.\\
     Per tant, les tonelades que s'han de enviar als abocadors han de ser estrictament major o igual a $0$ i menor o igual a $0.2$ vegades el nombre de tonelades transportades a l'incineradora:
     $$ 0 \leq T_{JK} \leq 0.2\cdot T_J = 0.2 \cdot T_{1J}+T_{2J} $$
     \item \textbf{Restriccions sobre els transports:}
     Cal recalcar que tot el tonelatge de residus que s'envien ha de ser igual als que són rebuts; és a dir:
     $$T_{IA} + T_{IB} = Res_I$$
     $$T_{JN} + T_{JS} = 0.2 \cdot T_J = 0.2 \cdot (T_{1J}+T_{2J})$$
 \end{itemize}
 \newpage
 \hspace{-1.5em}Finalment, podem expresar les restriccions del problema:
 \begin{enumerate}
     \item  $ 0 \leq T_{1A} \leq 500 $
     \item  $ 0 \leq T_{1B} \leq 500 $
     \item  $ 0 \leq T_{2A} \leq 400 $
     \item  $ 0 \leq T_{2B} \leq 400 $
     \item  $ T_{1A} + T_{1B} = 500 $
     \item  $ T_{2A} + T_{2B} = 400 $
     \item  $ 0 \leq T_{1A}+T_{2A} \leq 500 $
     \item  $ 0 \leq T_{1B}+T_{2B} \leq 600 $
     \item  $ 0 \leq T_{AN} \leq 0.2\cdot (T_{1A}+T_{2A}) $
     \item  $ 0 \leq T_{AS} \leq 0.2\cdot (T_{1A}+T_{2A}) $
     \item  $ 0 \leq T_{BN} \leq 0.2\cdot (T_{1B}+T_{2B}) $
     \item  $ 0 \leq T_{BS} \leq 0.2\cdot (T_{1B}+T_{2B}) $
     \item  $ T_{AN} + T_{AS} = 0.2 \cdot (T_{1A}+T_{2A}) $
     \item  $ T_{BN} + T_{BS} = 0.2 \cdot (T_{1B}+T_{2B}) $
\end{enumerate}
Algunes d'aquestes restriccions són redundants, pel que si eliminem les restriccinos que no són actives ens quedem amb:
\begin{enumerate}
     \item  $ 0 \leq T_{1A} $
     \item  $ 0 \leq T_{1B} $
     \item  $ 0 \leq T_{2A} $
     \item  $ 0 \leq T_{2B} $
     \item  $ T_{1A} + T_{1B} = 500 $
     \item  $ T_{2A} + T_{2B} = 400 $
     \item  $ T_{1A}+T_{2A} \leq 500 $
     \item  $ T_{1B}+T_{2B} \leq 600 $
     \item  $ 0 \leq T_{AN} $
     \item  $ 0 \leq T_{AS} $
     \item  $ 0 \leq T_{BN} $
     \item  $ 0 \leq T_{BS} $
     \item  $ T_{AN} + T_{AS} = 0.2 \cdot (T_{1A}+T_{2A}) $
     \item  $ T_{BN} + T_{BS} = 0.2 \cdot (T_{1B}+T_{2B}) $
\end{enumerate}
\newpage
\subsubsection{Apartat b)}
Transcribim el problema d'optimització (amb la funció de cost a minimitzar i les restriccions ja trobades a l'apartat \textit{(a)} ).\\
Per tal de millorar l'accessibilitat a les dades i la modificació de les mateixes, s'ha decidit fer el programa $glpk$ que resol el problema amb un fitxer apart on es guarden les dades necesaries per a resoldre'l.\\
\\
El resultat que hem obtingut a l'hora d'executar el nostre programa ha sigu de:
\begin{table}[h]
    \centering
    \begin{tabular}{ c | c }
        \textbf{Variables} & \textbf{Valor en Tonelades} \\ \hline
        $T_{1A}$ & 0 \\ \hline
        $T_{1B}$ & 500 \\ \hline
        $T_{2A}$ & 400 \\ \hline
        $T_{2B}$ & 0 \\ \hline
        $T_{AN}$ & 100 \\ \hline
        $T_{AS}$ & 0 \\ \hline
        $T_{BN}$ & 0 \\ \hline
        $T_{BS}$ & 80 \\
    \end{tabular}
\end{table}\\
És a dir, que hem de transportar tots el residus de la ciutat 1 a l'incineradora B i les seves cendres a l'abocador Sud.\\
Analogamnet per a la ciutat 2, cal transportat els seus residus a l'incineradora 2 i les cendrees generades a l'abocador Nord.\\
\\
A més, podem saber que el cost (valor optim/mínim del problema) serà de $107200$\euro.\\
\newpage
\subsubsection{Apartat c)}
Resolem ara cada una de les cuestions de l'apartat \textit{(c)}.
\begin{enumerate}
    \item \textbf{Com hauríem d’organitzar el transport?}\\
    Si apliquem la restricció obtenim la següent distribució del transport:
    \begin{table}[h]
    \centering
    \begin{tabular}{ c | c }
        \textbf{Variables} & \textbf{Valor en Tonelades} \\ \hline
        $T_{1A}$ & 300 \\ \hline
        $T_{1B}$ & 200 \\ \hline
        $T_{2A}$ & 100 \\ \hline
        $T_{2B}$ & 300 \\ \hline
        $T_{AN}$ & 60 \\ \hline
        $T_{AS}$ & 0 \\ \hline
        $T_{BN}$ & 0 \\ \hline
        $T_{BS}$ & 120 \\
    \end{tabular}
\end{table}\\
El cost d'aquesta ruta serà de: $115940$\euro.
    \item \textbf{Observant el fitxer de solució que es genera amb glpk (opció -o), de quina ruta
valdria la pena ampliar-ne la capacitat? \\
Quin seria el decreixement del cost per cada tona que s’ampliés?}
\begin{table}[h]
    \centering
    \begin{tabular}{ c | c | c | c | c | c | c }
        No. & Row name & St & Activity & Lower bound & Upper bound & Marginal \\\hline\hline
        1 & $T_{1A}$ & NU & $300$ & $0$ & $300$ &   \\ \hline 
        2 & $T_{1B}$ & B & $200$ & $0$ & $300$ & $-48$  \\\hline
        3 & $T_{2A}$ & B & $100$ & $0$ & $300$ &  \\\hline
        4 & $T_{2B}$ & NU & $300$ & $0$ & $300$ &  \\\hline
        5 & $T_{AN}$ & B & $80$ & $0$ &  & \\ \hline
        6 & $T_{AS}$ & NL & $0$ & $0$ &  & $9$ \\ \hline
        7 & $T_{BN}$ & NL & $0$ & $0$ &  & $9$ \\ \hline
        8 & $T_{BS}$ & B & $100$ & $0$ &  & \\
    \end{tabular}
    \caption{Taula extreta amb la comanda \texttt{-o} del $glpk$}
    \label{tab:my_label}
\end{table}\\
Observem mitjançant la taula que la variable que té el marginal més petit (i negatiu) és el $T_{1B}$, amb un marginal de $-48$ (és a dir, quant disminueix el cost per tona de residus transportada).\\
\\
Si s'amplies la ruta $T_{2B}$ en 200 tonelades més obtindrien un decreixement (teoric) de $98.56\cdot 200$\euro \hspace{0.06215em} sobre el valor obtingut en l'apartat \textit{(c1)}; és a dir, que tindrà un cost de $(115940-48\cdot 200)$\euro\hspace{0.06215em}$= (115940-9600)$\euro\hspace{0.06215em}$=106340$\euro.
    \item \textbf{Si calculem el cost per a aquesta ampliació de capacitat (només en una ruta), quadra amb el que observem al fitxer generat amb l’opció -o? Per què?}\\
    Si calculem el cost del transport dels residus amb la ruta ente la ciutat $2$ i l'incineradora $B$ ampliada, obtenim un cost total de $108060$\euro.\\
    Deduïm que l'aproximació feta pel margina (pendent del gradient de la funció) és prou bona.
\end{enumerate}
\newpage
\subsubsection{Apartat d)}
Relcalculant amb les noves dades, hem obtingut la següent resposta:\\
\begin{table}[h]
    \centering
    \begin{tabular}{ c | c }
        \textbf{Variables} & \textbf{Valor en Tonelades} \\ \hline
        $T_{1A}$ & 500 \\ \hline
        $T_{1B}$ & 0 \\ \hline
        $T_{2A}$ & 0 \\ \hline
        $T_{2B}$ & 400 \\ \hline
        $T_{AN}$ & 100 \\ \hline
        $T_{AS}$ & 0 \\ \hline
        $T_{BN}$ & 0 \\ \hline
        $T_{BS}$ & 80 \\
    \end{tabular}
\end{table}\\
És a dir, les rutes són les mateixes que en l'apartat \textit{(b)}, peró el cost d'aquesta ruta és de $53780$\euro\hspace{0.0625em}(es redueix a casi la meitat el cost trobar inicialment).


























\end{document}
