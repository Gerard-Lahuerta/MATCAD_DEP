\documentclass[a4paper, 11pt]{article}
\usepackage[utf8]{inputenc} 
\usepackage[T1]{fontenc}
\usepackage[catalan]{babel}
\usepackage{amsmath, amssymb, amsthm}
\usepackage[margin=1in]{geometry}
\usepackage{enumerate}
\usepackage{array}
\usepackage{graphicx}
\usepackage{wrapfig}
\usepackage{ragged2e} 
\usepackage{subfig}
\usepackage{caption}
\usepackage{subcaption}
\usepackage[dvipsnames]{xcolor}
\usepackage{float}
\usepackage{chngcntr}
\usepackage{ragged2e}
\usepackage{multirow}
\usepackage{vmargin}
\usepackage{hyperref}
\usepackage{url}
\usepackage{fancyhdr}
\usepackage{bigints}
\usepackage{listings}

\definecolor{codegreen}{rgb}{0,0.6,0}
\definecolor{codegray}{rgb}{0.5,0.5,0.5}
\definecolor{codepurple}{rgb}{0.58,0,0.82}
\definecolor{backcolour}{rgb}{0.95,0.95,0.92}

\lstdefinestyle{mystyle}{
    backgroundcolor=\color{white},   
    commentstyle=\color{codegreen},
    keywordstyle=\color{RoyalBlue},
    numberstyle=\tiny\color{codegray},
    stringstyle=\color{codepurple},
    basicstyle=\ttfamily\footnotesize,
    breakatwhitespace=false,         
    breaklines=true,                 
    captionpos=b,                    
    keepspaces=true,                 
    numbers=left,                    
    numbersep=5pt,                  
    showspaces=false,                
    showstringspaces=false,
    showtabs=false,                  
    tabsize=2
}

\lstset{style=mystyle}

\begin{document}

\begin{titlepage}
    \centering
    {\bfseries\LARGE Universitat Autònoma de Barcelona\newline Facultat de Ciències\par}
    \vspace{2cm}
    {\hspace{-1em}\includegraphics[width=0.4\textwidth]{logo.PNG}\par}
    \vspace{1cm}
    {\scshape\Huge Pràctica 2\par} %Fiat ut Nix --- et non erit nix
    \vspace{1cm}
    {\Large \itshape Authors: \par}
    {\Large \hspace{-1.75em} Gerard Lahuerta \& Ona Sánchez \par}
    {\Large 1601350 --- 1601181 \par}
    \vspace{1cm}
    {\Large 23 de Març del 2022\par}
\end{titlepage}

\justifying

\newpage
\setcounter{section}{-1}
\setcounter{page}{2}
\pagestyle{plain}
\tableofcontents
\cleardoublepage
\addcontentsline{}{chapter}
\newpage

\section{Introduction}
The data set we have chosen to work with is the EuStockMarkets. The data set contains the daily closing prices of major European stock indices: Germany DAX, Switzerland SMI, France CAC, and UK FTSE. The data are sampled in business time, i.e., weekends and holidays are omitted. \\
The data set has four columns: time (from 1991 to 1998), DAX, SMI, CAC and FTSE. It looks like this:
\begin{center}
    \begin{verbatim}
                DAX    SMI    CAC   FTSE
    1991.496 1628.75 1678.1 1772.8 2443.6
    1991.500 1613.63 1688.5 1750.5 2460.2
    1991.504 1606.51 1678.6 1718.0 2448.2
    1991.508 1621.04 1684.1 1708.1 2470.4
    1991.512 1618.16 1686.6 1723.1 2484.7
       ...    ...     ...    ...    ...
    \end{verbatim}
\end{center}
For this exercises, we will work with SMI, CAC and FTSE as the possible predictors for the best fitting model, and DAX will be the response variable.

\newpage
\section{Exercise 1}
\subsection{Statement}
Provide details of the chosen dataset. Design models to be analysed for this dataset.
\subsection{R-commands and Analysis}

\underline{Dataset initialization and details}:
\begin{lstlisting}[language=R]
library(datasets)
data("EuStockMarkets")
esm = data.frame(EuStockMarkets)
modelfull<-lm(DAX~., data = esm)
summary(modelfull)
\end{lstlisting}
\vspace{1em}
\underline{Output of the code}:\label{pocotexto}
\small\begin{verbatim}
Residuals:
    Min      1Q  Median      3Q     Max 
-335.26  -80.75    9.98   82.99  328.04 

Coefficients:
              Estimate Std. Error t value Pr(>|t|)    
(Intercept) -175.94567   44.66573  -3.939 8.48e-05 ***
SMI            0.49277    0.01532  32.163  < 2e-16 ***
CAC            0.49565    0.01544  32.105  < 2e-16 ***
FTSE          -0.01720    0.02089  -0.823     0.41    
---
Signif. codes:  0 ‘***’ 0.001 ‘**’ 0.01 ‘*’ 0.05 ‘.’ 0.1 ‘ ’ 1

Residual standard error: 109.4 on 1856 degrees of freedom
Multiple R-squared:  0.9898,	Adjusted R-squared:  0.9898 
F-statistic: 6.032e+04 on 3 and 1856 DF,  p-value: < 2.2e-16
\end{verbatim}\label{hawai}
\vspace{1em}
\underline{Analysis}:\\
We observe that without any changes to the model, it has a very small $p-value$ and a high\textit{ R-Squared} so we can deduce that the modifications that we will do to the model by changing the predictors will not be really significant, so there will not be a lot of differences between them (with the exception of the variable $FTSE$ that has a very high $p-value$ so we can't say anything about how it could affect future models). \\
\newpage
\hspace{-1.5em}\underline{Models to be analysed}:
\begin{lstlisting}[language=R]
Model = update(modelfull, .~.-SMI)
summary(Model)
Model = update(modelfull, .~.-CAC)
summary(Model)
Model = update(modelfull, .~.-FTSE)
summary(Model)

Model = update(modelfull, .~.-SMI-CAC)
summary(Model)
Model = update(modelfull, .~.-SMI-FTSE)
summary(Model)
Model = update(modelfull, .~.-FTSE-CAC)
summary(Model)
\end{lstlisting}
\vspace{1em}
\underline{Output of the code}:
\begingroup \fontsize{2pt}{4pt}\selectfont
\begin{verbatim}
Call:
lm(formula = DAX ~ CAC + FTSE, data = esm)

Residuals:
    Min      1Q  Median      3Q     Max 
-427.33  -79.59    7.40   89.52  389.70 

Coefficients:
              Estimate Std. Error t value Pr(>|t|)    
(Intercept) -1.575e+03  1.260e+01 -125.00   <2e-16 ***
CAC          8.479e-01  1.358e-02   62.46   <2e-16 ***
FTSE         6.218e-01  8.066e-03   77.09   <2e-16 ***
---
Signif. codes:  0 ‘***’ 0.001 ‘**’ 0.01 ‘*’ 0.05 ‘.’ 0.1 ‘ ’ 1

Residual standard error: 136.5 on 1857 degrees of freedom
Multiple R-squared:  0.9842,	Adjusted R-squared:  0.9842 
F-statistic: 5.78e+04 on 2 and 1857 DF,  p-value: < 2.2e-16

###############################################################

Call:
lm(formula = DAX ~ SMI + FTSE, data = esm)

Residuals:
    Min      1Q  Median      3Q     Max 
-319.20 -101.46   -9.06  109.06  566.57 

Coefficients:
             Estimate Std. Error t value Pr(>|t|)    
(Intercept) 885.87022   37.42814   23.67   <2e-16 ***
SMI           0.84173    0.01346   62.52   <2e-16 ***
FTSE         -0.33572    0.02292  -14.65   <2e-16 ***
---
Signif. codes:  0 ‘***’ 0.001 ‘**’ 0.01 ‘*’ 0.05 ‘.’ 0.1 ‘ ’ 1

Residual standard error: 136.4 on 1857 degrees of freedom
Multiple R-squared:  0.9842,	Adjusted R-squared:  0.9842 
F-statistic: 5.787e+04 on 2 and 1857 DF,  p-value: < 2.2e-16

###############################################################

Call:
lm(formula = DAX ~ SMI + CAC, data = esm)

Residuals:
    Min      1Q  Median      3Q     Max 
-336.83  -79.21   10.15   82.37  326.60 

Coefficients:
              Estimate Std. Error t value Pr(>|t|)    
(Intercept) -2.102e+02  1.617e+01  -13.01   <2e-16 ***
SMI          4.808e-01  4.741e-03  101.42   <2e-16 ***
CAC          5.017e-01  1.359e-02   36.93   <2e-16 ***
---
Signif. codes:  0 ‘***’ 0.001 ‘**’ 0.01 ‘*’ 0.05 ‘.’ 0.1 ‘ ’ 1

Residual standard error: 109.4 on 1857 degrees of freedom
Multiple R-squared:  0.9898,	Adjusted R-squared:  0.9898 
F-statistic: 9.05e+04 on 2 and 1857 DF,  p-value: < 2.2e-16

###############################################################

Call:
lm(formula = DAX ~ FTSE, data = esm)

Residuals:
    Min      1Q  Median      3Q     Max 
-408.43 -172.53  -45.71  137.68  989.96 

Coefficients:
              Estimate Std. Error t value Pr(>|t|)    
(Intercept) -1.331e+03  2.109e+01  -63.12   <2e-16 ***
FTSE         1.083e+00  5.705e-03  189.84   <2e-16 ***
---
Signif. codes:  0 ‘***’ 0.001 ‘**’ 0.01 ‘*’ 0.05 ‘.’ 0.1 ‘ ’ 1

Residual standard error: 240.3 on 1858 degrees of freedom
Multiple R-squared:  0.951,	Adjusted R-squared:  0.9509 
F-statistic: 3.604e+04 on 1 and 1858 DF,  p-value: < 2.2e-16

###############################################################

Call:
lm(formula = DAX ~ CAC, data = esm)

Residuals:
    Min      1Q  Median      3Q     Max 
-578.32 -250.24    2.49  245.09  548.50 

Coefficients:
              Estimate Std. Error t value Pr(>|t|)    
(Intercept) -1.493e+03  2.573e+01  -58.04   <2e-16 ***
CAC          1.806e+00  1.118e-02  161.62   <2e-16 ***
---
Signif. codes:  0 ‘***’ 0.001 ‘**’ 0.01 ‘*’ 0.05 ‘.’ 0.1 ‘ ’ 1

Residual standard error: 279.6 on 1858 degrees of freedom
Multiple R-squared:  0.9336,	Adjusted R-squared:  0.9336 
F-statistic: 2.612e+04 on 1 and 1858 DF,  p-value: < 2.2e-16

###############################################################

Call:
lm(formula = DAX ~ SMI, data = esm)

Residuals:
    Min      1Q  Median      3Q     Max 
-285.75 -106.88  -20.15  104.20  603.45 

Coefficients:
             Estimate Std. Error t value Pr(>|t|)    
(Intercept) 3.478e+02  7.558e+00   46.02   <2e-16 ***
SMI         6.465e-01  2.008e-03  321.91   <2e-16 ***
---
Signif. codes:  0 ‘***’ 0.001 ‘**’ 0.01 ‘*’ 0.05 ‘.’ 0.1 ‘ ’ 1

Residual standard error: 144 on 1858 degrees of freedom
Multiple R-squared:  0.9824,	Adjusted R-squared:  0.9824 
F-statistic: 1.036e+05 on 1 and 1858 DF,  p-value: < 2.2e-16
\end{verbatim}\label{muchotexto}\endgroup
\vspace{1em}
\hspace{-1.5em}\underline{Analysis}:\\
As we can see in the outputs, the value of the $p-value$ does not change depending on which predictors we use, it always stays in $< 2.2e-16$. Instead, the \textit{R-squared error} varies between $0.9336$ (when the only predictor is CAC) and $0.9898$ (when we use all the predictors).\\
We also have to consider the full model, analysed in section \textcolor{blue}{\ref{hawai}} as a model to be analysed.
\newpage
\section{Exercise 2}
\subsection{Statement}
Apply backward selection to find the best fit model using p-value and AIC
criteria. Compare the results found by both methods. Do the same for
forward selection. Comment on the results.
\subsection{R-commands and Analysis}
\underline{Backward selection using p-value}:
\begin{lstlisting}[language=R]
library(MASS)

Model=update(modelfull, .~.-SMI)
summary(Model)
Model=update(modelfull, .~.-CAC)
summary(Model)
Model=update(modelfull, .~.-FTSE)
summary(Model)
Model=modelfull
summary(Model)
\end{lstlisting}
The output of the first three models have already been explained in section \textcolor{blue}{\ref{muchotexto}}.\\
The output of the last model has already been explained in section \textcolor{blue}{\ref{pocotexto}}.\\
\vspace{2em}
\\\underline{Backward selection using AIC criteria}:
\begin{lstlisting}[language=R]
modelbackward = stepAIC(modelfull, trace = TRUE, direction = "backward")
\end{lstlisting}
\vspace{1em}
\underline{Output of the code}:
\small\begin{verbatim}
Start:  AIC=17469.1
DAX ~ SMI + CAC + FTSE

       Df Sum of Sq      RSS   AIC
- FTSE  1      8113 22217334 17468
<none>              22209221 17469
- CAC   1  12333661 34542881 18289
- SMI   1  12378364 34587584 18291

Step:  AIC=17467.78
DAX ~ SMI + CAC

       Df Sum of Sq       RSS   AIC
<none>               22217334 17468
- CAC   1  16315505  38532840 18490
- SMI   1 123050941 145268275 20958
\end{verbatim}
\newpage
\hspace{-1.5em}\underline{Forward selection using p-value}:
\begin{lstlisting}[language=R]
colnames(esm)
modelnull = lm(DAX~1, data = EuStockMarkets)
summary(modelnull)
Model = update(modelnull, .~.+SMI)
summary(Model)
Model = update(modelnull, .~.+CAC)
summary(Model)
Model = update(modelnull, .~.+FTSE)
summary(Model)
Model = update(modelnull, .~.+SMI+CAC)
summary(Model)
Model = update(modelnull, .~.+SMI+FTSE)
summary(Model)
Model = update(modelnull, .~.+SMI+CAC+FTSE)
summary(Model)
\end{lstlisting}
The outputs of these models have already been explained in \textcolor{blue}{\ref{muchotexto}}.
\vspace{2em}
\\\underline{Forward selection using AIC criteria}:
\begin{lstlisting}[language=R]
modelfull = formula(DAX~SMI+CAC+FTSE)
modelnull = lm(DAX~1, data=esm)
modelforward = stepAIC(modelnull, trace = TRUE, direction = "forward",                         scope = modelfull)
\end{lstlisting}
\vspace{1em}
\underline{Output of the code}:
\small\begin{verbatim}
Start:  AIC=26000.62
DAX ~ 1

       Df  Sum of Sq        RSS   AIC
+ SMI   1 2149092423   38532840 18490
+ FTSE  1 2080369991  107255272 20394
+ CAC   1 2042356988  145268275 20958
<none>               2187625263 26001

Step:  AIC=18489.96
DAX ~ SMI

       Df Sum of Sq      RSS   AIC
+ CAC   1  16315505 22217334 17468
+ FTSE  1   3989958 34542881 18289
<none>              38532840 18490

Step:  AIC=17467.78
DAX ~ SMI + CAC

       Df Sum of Sq      RSS   AIC
<none>              22217334 17468
+ FTSE  1    8113.4 22209221 17469
\end{verbatim}
\newpage
\hspace{-1.5em}\underline{Analysis}:\\
For the backward selection using $p-value$, we start with the full model, and we must remove the element with the highest $p-value$ each time. As we can see in the outputs, all the models have the same $p-value$ ($2.2\cdot 10^{-16}$) so we can not remove any element, staying with the full model.\\\\
Moreover, for the forward selection (also using $p-value$), we start with no variables and we add the one with the lowest $p-value$ each time. As in the backward selection, we finally add all the variables because the $p-value$ in all the models are the same ($2.2\cdot 10^{-16}$).\\\\
When we use the AIC criteria, for both cases (backward and forward) the result is the same: we must stay with the three predictors, using the full model.


\newpage
\section{Exercise 3}
\subsection{Statement}
 Find the best possible subset of variables to select the best fit model.
Compare the results with the final models obtained in the previous point.
\subsection{R-commands and Analysis}
\begin{lstlisting}[language=R]
library(olsrr)
Model = lm(DAX~SMI+CAC+FTSE, data = EuStockMarkets)
olsstepallpossible(Model)

\end{lstlisting}
\underline{Output of the code}:
\small\begin{verbatim}
Index N   Predictors  R-Square Adj. R-Square  Mallow's Cp
1     1 1          SMI 0.9823860     0.9823765  1364.146757
3     2 1         FTSE 0.9509718     0.9509454  7107.204410
2     3 1          CAC 0.9335954     0.9335597 10283.909009
4     4 2      SMI CAC 0.9898441     0.9898331     2.678025
5     5 2     SMI FTSE 0.9842099     0.9841929  1032.710373
6     6 2     CAC FTSE 0.9841894     0.9841724  1036.446149
7     7 3 SMI CAC FTSE 0.9898478     0.9898314     4.000000
\end{verbatim}
\begin{lstlisting}[language=R]
olsstepbestsubset(Model)
\end{lstlisting}
\vspace{1em}
\underline{Output of the code}:
\begingroup \fontsize{2pt}{4pt}\selectfont
\begin{verbatim}
Best Subsets Regression  
---------------------------
Model Index    Predictors
---------------------------
     1         SMI          
     2         SMI CAC      
     3         SMI CAC FTSE 
---------------------------

                                                             Subsets Regression Summary                                                             
----------------------------------------------------------------------------------------------------------------------------------------------------
                       Adj.        Pred                                                                                                              
Model    R-Square    R-Square    R-Square      C(p)          AIC           SBIC          SBC            MSEP            FPE          HSP       APC  
----------------------------------------------------------------------------------------------------------------------------------------------------
  1        0.9824      0.9824      0.9823    1364.1468    23770.4141    18489.9164    23786.9991    38574317.3445    20761.1802    11.1679    0.0177 
  2        0.9898      0.9898      0.9898       2.6780    22748.2273    17469.7867    22770.3406    22253232.8989    11983.3972     6.4462    0.0102 
  3        0.9898      0.9898      0.9898       4.0000    22749.5479    17471.1138    22777.1896    22257098.3979    11991.9087     6.4508    0.0102 
----------------------------------------------------------------------------------------------------------------------------------------------------
AIC: Akaike Information Criteria 
 SBIC: Sawa's Bayesian Information Criteria 
 SBC: Schwarz Bayesian Criteria 
 MSEP: Estimated error of prediction, assuming multivariate normality 
 FPE: Final Prediction Error 
 HSP: Hocking's Sp 
 APC: Amemiya Prediction Criteria 
\end{verbatim}
\endgroup

\begin{lstlisting}[language=R]
library(leaps)
modelsubsets = regsubsets(DAX~SMI+CAC+FTSE, data=EuStockMarkets, nbest=2)
summary(modelsubsets)$which
\end{lstlisting}
\vspace{1em}
\underline{Output of the code}:
\small\begin{verbatim}
     (Intercept)   SMI   CAC  FTSE
1        TRUE  TRUE FALSE FALSE
1        TRUE FALSE FALSE  TRUE
2        TRUE  TRUE  TRUE FALSE
2        TRUE  TRUE FALSE  TRUE
3        TRUE  TRUE  TRUE  TRUE
\end{verbatim}
\vspace{1em}
\underline{Analysis}:\\
We conclude (with the the outputs above) that our best variable sample are all the variables of the dataset ($SMI, CAC, FTSE$), the same conclusion that we obtained in exercise $2$.\\
This is because of many things (for example): it has one of the lowest \textit{R-Squared error}, it has the most accurate Mallow's CP (number of predictors plus the intercept; indicates that the model produces relatively precise and unbiased estimates) and its Final Prediction Error is quite good.\\
The last output shows all the possible variable subset selection done by $R$ (the last one is the best).
\end{document}
